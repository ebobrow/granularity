\documentclass[12pt]{article}
\usepackage{amssymb, amsmath, amsfonts, amsthm, graphicx, tcolorbox}
\usepackage{enumitem, bussproofs, cmll}
\parskip = .2in
\parindent = 0in

\title{Assertion Level Rules in Tennant's System $\mathbb{C}$}
\author{Elliot Bobrow}
\date{\today}

\begin{document}

\maketitle

\tableofcontents

\section{Introduction}

Neil Tennant defines a system $\mathbb{C}$ which enforces relevance and removes
the Ex Falso rule. Some other bozos introduce a deranged system to generate
Sequent Calculus rules from assertions. Let's combine them.

\section{Doing it}

We start with the simple assertion $\forall n.\forall m.n<m\rightarrow
S(n)<S(m)$. Following the rules outlined in Atomic Metadeduction, we first
exhaustively apply the rules of our calculus to obtain every possible tree (of
which there is one):
% \[
%     \frac{\frac{H_{1A}\vdash n<m\quad H_{1b},S(n)<S(m)\vdash
%     H_2}{H_{1A},H_{1B},n<m\rightarrow S(n)<S(m)\vdash H_2}\rightarrow
%     L}{H_{1A},H_{1B},\forall n.\forall m.n<m\rightarrow S(n)<S(m)\vdash
%     H_2}\forall L' (\times 2)
% \]
\begin{prooftree}
    \AxiomC{$H_{1A}\vdash n<m$}
    \AxiomC{$H_{1B},S(n)<S(m)\vdash H_2$}
    \RightLabel{$\rightarrow L$}
    \BinaryInfC{$H_{1A},H_{1B},n<m\rightarrow S(n)<S(m)\vdash H_2$}
    \RightLabel{$\forall L'(\times 2)$}
    \UnaryInfC{$H_{1A},H_{1B},\forall n.\forall m.n<m\rightarrow S(n)<S(m)\vdash
    H_2$}
\end{prooftree}

This uses my $\forall L'$ rule
\begin{prooftree}
    \AxiomC{$\Delta,\psi^x:\theta$}
    \RightLabel{$\forall L'\quad\quad\text{$x$ closed term}$}
    \UnaryInfC{$\Delta,\forall x\psi(x):\theta$}
\end{prooftree}
to replace Tennant's
\begin{prooftree}
    \AxiomC{$\Delta,\psi^x_{t_1},\dots,\psi^x_{t_n}:\theta$}
    \RightLabel{$\forall L$}
    \UnaryInfC{$\Delta,\forall x\psi(x)$}
\end{prooftree}
in line with the fact that AM removes the CONTR rule when deriving these
assertion level rules.

So anyways now there are three possible rules. By instantiating $H_{1A}=n<m$ we
get
\[
    \frac{\Gamma,S(n)<S(m)\vdash\Delta}{\Gamma,n<m\vdash\Delta}.
\]
This removes the $n<m$ from the assumptions of the top sequent in the rule in
AM. I think that's good because you probably just want to use one or the other.
By instantiating $H_2=S(n)<S(m)$ we get
\[
    \frac{\Gamma\vdash n<m,\Delta}{\Gamma\vdash S(n)<S(m),\Delta}.
\]
This seems good. Is it the R version of the above. We probably dont want
$\Delta$ but we do need $\Gamma$? (Interesting--this time
identical to the one in AM). Finally, if we instantiate both,
\[
    \frac{}{\Gamma,n<m\vdash S(n)<S(m),\Delta}.
\]
This time we definitely don't want $\Gamma$ and $\Delta$.

\section{Experimenting with $\exists$}

Now let's try $\forall X\forall Y.X\star Y\rightarrow\exists a.(a\in X)\wedge
(a\in Y)$ where $\star$ is a symbol that Prof Stone thinks should exist. Start
the tree:
\[
    \frac{H_{1A}\vdash X\star Y\quad H_{1B},\exists a.(a\in X)\wedge (a\in
    Y)\vdash H_2}{H_1,X\star Y\rightarrow\exists a.(a\in X)\wedge(a\in
    Y)\vdash H_2}\rightarrow L
\]
Here we reach a crossroads. In AM, there is no $\exists$ so we replace $\exists
x.F$ with $\neg\forall x.\neg F$. If we do that, because $\mathbb{C}$ has no Ex
Falso rule, we are stuck unless $H_2=\bot$. Let's try stopping here and
producing rules.
\[
    \frac{\Gamma,\exists a.(a\in X)\wedge(a\in Y)\vdash\Delta}{\Gamma,X\star
    Y\vdash\Delta}\quad \frac{\Gamma\vdash X\star Y,\Delta}{\Gamma\vdash\exists
    (\dots)}\quad \frac{}{\Gamma,X\star
    Y\vdash\exists(\dots),\Delta}
\]
Okay these are just like the ones we just did. But the point is to get rid of
existentials!  What if now we try applying Tennant's $\exists L$ rule but NOT
his $\wedge L$ rule:
\begin{prooftree}
    \AxiomC{$H_{1A}\vdash X\star Y$}
                                    % \AxiomC{$H_{1BA}\vdash a\in X,H_2$}
                                    % \AxiomC{$H_{1BB}\vdash a\in Y,H_2$}
                                    % \RightLabel{$\wedge R$}
                                    \AxiomC{$H_{1B},(a\in X)\wedge(a\in Y)\vdash H_2$}
                                    \RightLabel{$\exists L$}
                                    \UnaryInfC{$H_{1B},\exists a.(a\in X)\wedge(a\in Y)\vdash H_2$}
    \RightLabel{$\rightarrow L$}
    \BinaryInfC{$H_1,X\star Y\rightarrow\exists a.(a\in X)\wedge(a\in Y)\vdash
    H_2$}
\end{prooftree}
% Note that the actual $\mathbb{C}$ deals only with single conclusion subsequents
% but we allow $H_2$ to come along for the ride.

With $H_{1A}=X\star Y$:
\[
    \frac{H_{1},(a\in X)\wedge(a\in Y)\vdash H_2}{X\star
    Y,H_{1}\vdash H_2}
\]
note with $\wedge L$ we could product incorrect rules here to do with the lack
of Skolem whatever. we NEED both requirements on $a$.

With $H_2=a\in X\wedge a\in Y$:
\[
    \frac{H_{1}\vdash X\star Y}{H_{1}\vdash a\in X\wedge a\in Y}
\]
note with $\wedge L$ we could also have the above but with only half in the
bottom. not incorrect but a little silly

With both:
\[
\frac{}{X\star Y\vdash a\in X\wedge a\in Y}
\]

So in the cases where we instantiate with the $a$ stuff we do want a Skolen
fuction or something to show that it's not ALL $a$. Or we should have just
stopped before applying $\exists R$. In other words we should leave the
$\exists$ if is occurs in the bottom of our new rule. Algorithm could be
instantiate $H_1$ and $H_2$ and then apply AXIOM and THEN apply $\exists$ rules.

What happens if we run this assertion through the regular rules? If we treat
$\exists a.\cdots$ as $\neg\forall a.\neg\cdots$ and don't use $\wedge R$
becaues it isn't defined in AM, then we get the rules
\[
    \frac{\Gamma,X\star Y,a(X,Y)\in X\wedge a(X,Y)\in Y\vdash\Delta}{\Gamma,A\star Y\vdash\Delta}
\]

\[
    \frac{\Gamma\vdash X\star Y,\Delta}{\Gamma\vdash a(X,Y)\in X\wedge a(X,Y)\in
    Y,\Delta}
\]

\[
    \frac{}{\Gamma,X\star Y\vdash a(X,Y)\in X\wedge a(X,Y)\in Y,\Delta}
\]
where $a(X,Y)$ is a Skolem function which we just assume to be the correct
function? This rule could not be automated I think.

\section{Same rule but flip the arrow}

\begin{prooftree}
    % \AxiomC{$H_{1BA}\vdash a\in X,H_2$}
    % \AxiomC{$H_{1BB}\vdash a\in Y,H_2$}
    % \RightLabel{$\wedge R$}
    \AxiomC{$H_{1B}\vdash (a\in X)\wedge(a\in Y)$}
    \RightLabel{$\exists R$}
    \UnaryInfC{$H_{1A}\vdash \exists a.(a\in X)\wedge(a\in Y)$}
                                    \AxiomC{$H_{1B},X\star Y\vdash H_2$}
    \RightLabel{$\rightarrow L$}
    \BinaryInfC{$H_1,\exists a.(a\in X)\wedge(a\in Y)\rightarrow X\star Y\vdash H_2$}
\end{prooftree}
Gives rules
\[
\frac{H_1,X\star Y\vdash H_2}{H_1,(a\in X)\wedge(a\in Y)\vdash H_2}\quad\quad
\frac{H_1\vdash (a\in X)\wedge(a\in Y)}{H_1\vdash X\star Y}
\]

\[
    \frac{}{(a\in X)\wedge(a\in Y)\vdash X\star Y}
\]
These work, again because we skipped the $\wedge R$ application. Almost like we
should stop applying rules when we reach an $\exists$ so that all the
requirements stay together? As long as the introduced variable is unused before
the application of the rule.

\section{Trying this with Linear Logic!}

We try the silly assertion $A\oplus B\multimap C\otimes D$.
\begin{prooftree}
    \AxiomC{$H_{1AA}\vdash A$}
    \AxiomC{$H_{1AB}\vdash B$}
    \RightLabel{$\otimes R$}
    \BinaryInfC{$H_{1A}\vdash A\otimes B$}

                                            \AxiomC{$H_{1B},C\vdash H_2$}
                                            \AxiomC{$H_{1B},D\vdash H_2$}
                                            \RightLabel{$\oplus L$}
                                            \BinaryInfC{$H_{1A},C\oplus D\vdash H2$}
    \RightLabel{$\multimap L$}
    \BinaryInfC{$H1,A\oplus B\multimap C\otimes D\vdash H2$}
\end{prooftree}
If we set $H_{1AA}=A$ and DON'T instantiate the other $H$s (this diverges from
the given algorithm), we get the rule
\[
    \frac{H_{1AB}\vdash B\quad H_{1B},C\vdash H_2\quad H_{1B},D\vdash
    H_2}{A,H_{1AB},H_{1B}\vdash H_2}.
\]
The reason we want to do it like this is that the linear logic rules build in
how resources are used, and we want to keep that information instead of
uniformly tacking on a $\Gamma$ to all leaves.

The rest of the rules are as follows. Doing $H_{1AA}=A$ and $H_{1AB}=B$ gives
\[
    \frac{H_{1B},C\vdash H_2\quad H_{1B},D\vdash H_2}{A,B,H_{1B}\vdash H_2}.
\]
Doing $H_2=C$:
\[
    \frac{H_{1AA}\vdash A\quad H_{1AB}\vdash B\quad H_{1B},D\vdash
    C}{H_{1AA},H_{1AB},H_{1B}\vdash C}.
\]
Doing $H_2=C,D$:
\[
    \frac{H_{1AA}\vdash A\quad H_{1AB}\vdash B}{H_{1AA},H_{1AB},H_{1B}\vdash C,D}.
\]
Uh oh! We don't want $H_{1B}$. So when we instantiate a leaf we also need to
check if there are new unused assumptions, in which case we get rid of them.

Here is my new algorithm for linear logic: Starting with the sequent
$\Gamma,F\vdash\Delta$, exhaustively apply the rules to generate all possible
trees (splitting $\Gamma$ and $\Delta$ as needed). Then, determine
instantiations of $\Gamma$ and $\Delta$ as follows. We want to enable one or
more applications of the AXIOM rule on leaves. Instantiate a $\Gamma$ or
$\Delta$ fragment with an atom from the other side of the turnstyle. IMPORTANT:
If you are instantiating $\Delta$, you must use ALL atoms from the other side of
the turnstyle. Then, remove any unused assumptions or conclusions from the leaf.
Also, if it is a fragment that is present in multiple leaves, you may need to
instantiate it with all atoms from both leaves.

Then, assemble the rule with all unused leaves on the top (keeping info about
$\Gamma$ and $\Delta$ fragments) and the bottom with the reassembled
$\Gamma\vdash\Delta$.

I applied this process to the following:
\begin{align*}
    A\otimes B &\multimap C \\
    A\oplus B &\multimap C \\
    A\& B &\multimap C \\
    A\parr B &\multimap C \\
    A &\multimap C\otimes D \\
    A &\multimap C\oplus D \\
    A &\multimap C\& D \\
    A &\multimap C\parr D
\end{align*}
Maybe I'll type them up. They're in my camera roll and look reasonable.

\section{Can we prove some stuff?}

Soundness of AM: For all sets of formulas Th, all formulas F and all proofs of
$\Gamma\vdash_{Th,F}\Delta$ there exists a proof for $\Gamma,F\vdash_{Th}\Delta$
without $F$-rules.

Their proof involves the use of CUT, which we don't have in $\mathbb{C}$.

TODO

Next steps:
\begin{enumerate}
    \item Try proving some stuff
    \item Try more things in $\mathbb{C}$
    \item Read about Skolem, figure out why AM uses Skolem functions at all,
        read about the even more liberated $\delta$-rule
    \item Organize this doc
\end{enumerate}

\section{I made a mistake?}

I don't know what the fuck happened today but I guess I've been doing it wrong
the whole time? Look:
\begin{prooftree}
    \AxiomC{$H_1\vdash A,H_2$}
        \AxiomC{$H_1,B\vdash H_2$}
    \RightLabel{$\rightarrow L$}
    \BinaryInfC{$H_1,A\rightarrow B\vdash H_2$}
\end{prooftree}
Now instantiate $H_1=A$:
\[
\frac{\Gamma,A,B\vdash\Delta}{\Gamma,A\vdash\Delta}.
\]
That's what we were getting before. But if we instantiate $H_2=B$:
\[
\frac{\Gamma\vdash A,B,\Delta}{\Gamma\vdash B,\Delta}.
\]
We shouldn't have $B$ in the top sequent!! Was I doing something different
before or just fell asleep and didn't notice because their rules don't have $B$
in the top sequent??

Now try using Tennant's rule:
\begin{prooftree}
    \AxiomC{$H_{1A}\vdash A$}
        \AxiomC{$H_{1B},B\vdash H_2$}
    \RightLabel{$\rightarrow L$}
    \BinaryInfC{$H_{1A},H_{1B},A\rightarrow B\vdash H_2$}
\end{prooftree}
and the slightly revised algorithm as follows: Instantiate some of the
hypotheses as in the MD paper. However, for uninstantiated hypotheses we want to
leave them instead of instantiating them as empty. The only exception is if we
apply AXIOM to a leaf and one of the hypotheses is unused---then for relevance,
it must be empty (for example, $H_{1B}$ if we instantiave $H_2=B$). Then
reassemble the rule but do not tack on $\Gamma$ and $\Delta$

Gives rules:
\[
    \frac{H_{1B},B\vdash H_2}{H_{1B},A\vdash H_2}
    \quad\quad\quad\quad
    \frac{H_{1A}\vdash A}{H_{1A}\vdash B}
    \quad\quad\quad\quad
    \frac{}{A\vdash B}
\]
These actually look pretty good.

In summary, the transition from MD to $\mathbb{C}$ is
\[
\frac{\Gamma,A,B\vdash\Delta}{\Gamma,A\vdash\Delta}\longrightarrow
    \frac{\Gamma,B\vdash \Delta}{\Gamma,A\vdash \Delta}
\]

\[
\frac{\Gamma\vdash A,B,\Delta}{\Gamma\vdash B,\Delta}\longrightarrow
    \frac{\Gamma\vdash A}{\Gamma\vdash B}
\]

\[
\frac{}{\Gamma,A\vdash B,\Delta}\longrightarrow
\frac{}{A\vdash B}
\]

\section{Playing more with quantifiers}

Try this algebra thing:
\begin{prooftree}
    \AxiomC{$H_1\vdash f()^2=1,H_2$}
    \RightLabel{$\forall R$}
    \UnaryInfC{$H_1\vdash \forall x.x^2=1,H_2$}
            \AxiomC{$H_1,xy=yx\vdash H_2$}
            \RightLabel{$\forall L (\times 2)$}
            \UnaryInfC{$H_1,\forall x.\forall y.xy=yx$}
    \RightLabel{$\rightarrow L$}
    \BinaryInfC{$H_1,(\forall x.x^2=1)\rightarrow (\forall x.\forall y.xy=yx)\vdash H_2$}
\end{prooftree} 
where $f()$ is a Skolem function of 0 variables. For this trivial example, $f()$
is just a constant so maybe it doesn't really make sense that it has to be a
Skolem function.

Now try this:
\begin{prooftree}
    \AxiomC{$H_1\vdash P(x,f(x)),H_2$}
    \RightLabel{$\forall R$}
    \UnaryInfC{$H_1\vdash \forall y.P(x,y),H_2$}
        \AxiomC{$H_1,Q(x)\vdash H_2$}
    \RightLabel{$\rightarrow L$}
    \BinaryInfC{$H_1,(\forall y.P(x,y))\rightarrow Q(x)\vdash H_2$}
\end{prooftree}
This gives
\[
    \frac{\Gamma,P(x,f(x)),Q(x)\vdash\Delta}{\Gamma,P(x,f(x))\vdash\Delta}
    \quad\quad
    \frac{\Gamma\vdash P(x,f(x)),Q(x),\Delta}{\Gamma\vdash P(x,f(x)),\Delta}
\]

\[
    \frac{}{\Gamma,P(x,f(x))\vdash Q(x),\Delta}
\]
I still don't understand why $y$ should be a function of $x$. Based on the
original assertion, it should hold for all $y$.

Let's derive some rules for $\exists$ in MD. First we derive a rule for $\neg$,
which we unwrap as $\neg P\equiv P\rightarrow\bot$. The left rule is
\begin{prooftree}
    \AxiomC{$\Gamma\vdash P,\Delta$}
            \AxiomC{}
            \UnaryInfC{$\Gamma,\bot\vdash\Delta$}
    \RightLabel{$\rightarrow L$}
    \BinaryInfC{$\Gamma,P\rightarrow\bot\vdash\Delta$}
\end{prooftree}
and the right rule is
\begin{prooftree}
    \AxiomC{$\Gamma,P\vdash\bot,\Delta$}
    \RightLabel{$\rightarrow R$}
    \UnaryInfC{$\Gamma\vdash P\rightarrow\bot,\Delta$}
\end{prooftree}
Now we can unwrap $\exists x.P(x)$ as $\neg(\forall x.\neg P(x))$. The left rule
is
\begin{prooftree}
    \AxiomC{$\Gamma,P(f(\dots))\vdash \Delta$}
    \RightLabel{$\neg R$}
    \UnaryInfC{$\Gamma\vdash \neg P(f(\dots)),\Delta$}
    \RightLabel{$\forall R$}
    \UnaryInfC{$\Gamma\vdash\forall x.\neg P(x),\Delta$}
    \RightLabel{$\neg L$}
    \UnaryInfC{$\Gamma,\neg(\forall x.\neg P(x))\vdash\Delta$}
\end{prooftree}
where $f$ is the Skolem function and the right rule is
\begin{prooftree}
    \AxiomC{$\Gamma\vdash P(x),\Delta$}
    \RightLabel{$\neg L$}
    \UnaryInfC{$\Gamma,\neg P(x)\vdash \Delta$}
    \RightLabel{$\forall L\quad\quad\text{where $H$ is closed}$}
    \UnaryInfC{$\Gamma,\forall x.\neg P(x)\vdash \Delta$}
    \RightLabel{$\neg R$}
    \UnaryInfC{$\Gamma\vdash\neg(\forall x.\neg P(x)),\Delta$}
\end{prooftree}

\section{Playing with quantifiers in Superdeduction}

Atomic Metadeduction is based on a system called Superdeduction. It is
restricted to theorems of the form $\forall
\overline{x}.(P\Leftrightarrow\varphi)$ where $P$ is an atomic proposition.

What if we want to generate rules for this silly theorem: $\forall G.(\exists
H.G\cong H)\Leftrightarrow(\exists H.H\cong G)$. It is not in the proper form,
so first we would have to Skolemize it:
\[
    \forall G,f(G).G\cong f(G)\Leftrightarrow(\exists H.H\cong G).
\]
Now we generate the left rule with the following tree:
\begin{prooftree}
    \AxiomC{$\Gamma,H\cong G\vdash\Delta$}
    \RightLabel{$\exists L\quad\quad H\notin FV(\Gamma,\Delta)$}
    \UnaryInfC{$\Gamma,\exists H.H\cong G\vdash\Delta$}
\end{prooftree}
We collect premises and side conditions and replace the righthand side of our
theorem with the left to get the rule:
\begin{prooftree}
    \AxiomC{$\Gamma,H\cong G\vdash\Delta$}
    \RightLabel{$\cong L\quad\quad H\notin FV(\Gamma,\Delta)$}
    \UnaryInfC{$\Gamma,G\cong f(G)\vdash\Delta$}
\end{prooftree}
where $H$ is a variable.
The right rule goes similarly:
\begin{prooftree}
    \AxiomC{$\Gamma\vdash H\cong G,\Delta$}
    \RightLabel{$\exists R$}
    \UnaryInfC{$\Gamma\vdash\exists H.H\cong G,\Delta$}
\end{prooftree}
We collect premises and side conditions and replace the righthand side of our
theorem with the left to get the rule:
\begin{prooftree}
    \AxiomC{$\Gamma\vdash H\cong G,\Delta$}
    \RightLabel{$\cong R$}
    \UnaryInfC{$\Gamma\vdash G\cong f(G),\Delta$}
\end{prooftree}
where $H$ is any term.
I don't have a good intuition for these but I also don't have a good intuition
for the regular sequent calculus quantifier rules.

Also note that we don't have an axiom rule like we get in AM, but that's okay
because we can just use the left rule and then we get to apply axiom. In AM, the
same statement (only considering the forward direction because I am lazy) gives
the rules, for any closed $H$,
\[
    \frac{\Gamma,G\cong H,f(G)\cong H\vdash\Delta}{\Gamma,G\cong H\vdash\Delta}
    \quad
    \frac{\Gamma\vdash G\cong H,f(G)\cong G\Delta}{\Gamma\vdash,f(G)\cong G\Delta}
    \quad
    \frac{}{\Gamma,G\cong H\vdash f(G)\cong G,\Delta}
\]
where again the third one is derivable from the first. Interestingly, we lose
the restrictions on $H$ being fresh in the sequent.

Given an arbitrary proposition $\psi\Leftrightarrow\varphi$, can we convert it
to the necessary form for superdeduction? We can pull quantifiers out of $\psi$
using the equivalences
\begin{align*}
    (\forall x\psi)\rightarrow\varphi &\equiv \exists x(\psi\rightarrow\varphi) \\
    (\exists x\psi)\rightarrow\varphi &\equiv \forall x(\psi\rightarrow\varphi)
\end{align*}
and then Skolemize to remove $\exists$. I can't think of a way to convert
something like $A\vee B\Leftrightarrow C$ into proper form.

\section{Trying a real proof (in $\mathbb{C}$)}

We wish to prove the theorem $\forall X.\forall Y.X\subseteq Y\rightarrow \forall Z.X\cup Z\subseteq
Y\cup Z$. This requires the generation of inference rules for $\subseteq$ and
$\cup$.

\subsection{Generating rules using Atomic Metadeduction}
First, subset:
\begin{prooftree}
    \AxiomC{$\Gamma_1\vdash X\subseteq Y$}
        \AxiomC{$\Gamma_2\vdash a\in X$}
            \AxiomC{$\Gamma_3,a\in Y\vdash\Delta$}
        \RightLabel{$\rightarrow L$}
        \BinaryInfC{$\Gamma_2,\Gamma_3,a\in X\rightarrow a\in Y\vdash\Delta$}
        \RightLabel{$\forall L'\quad\quad\text{$a$ closed}$}
        \UnaryInfC{$\Gamma_2,\Gamma_3,\forall a.a\in X\rightarrow a\in
        Y\vdash\Delta$}
    \RightLabel{$\rightarrow L$}
    \BinaryInfC{$\Gamma_1,\Gamma_2,\Gamma_3,X\subseteq Y\rightarrow (\forall
    a.a\in X\rightarrow a\in Y)\vdash \Delta$}
\end{prooftree}
Now I will type up every possible rule using my revised algorithm.
\[
\frac{\Gamma_2\vdash a\in X\quad\Gamma_3,a\in Y\vdash\Delta}
     {X\subseteq Y,\Gamma_2\Gamma_3\vdash\Delta}\subseteq 1
\quad\quad
\frac{\Gamma_1\vdash X\subseteq Y\quad\Gamma_3,a\in Y\vdash\Delta}
     {\Gamma_1\Gamma_3,a\in X\vdash\Delta}\subseteq 2
\]

\[
\frac{\Gamma_1\vdash X\subseteq Y\quad\Gamma_2\vdash a\in X}
     {\Gamma_1,\Gamma_2\vdash a\in Y}\subseteq 3
\quad\quad
\frac{a\in Y,\Gamma_3\vdash\Delta}
     {X\subseteq Y,a\in X,\Gamma_3\vdash\Delta}\subseteq 4
\]

\[
\frac{\Gamma_2\vdash a\in X}
     {X\subseteq Y,\Gamma_2\vdash a\in Y}\subseteq 5
\quad\quad
\frac{\Gamma_1\vdash X\subseteq Y}
     {\Gamma_1,a\in X\vdash a\in Y}\subseteq 6
\quad\quad
\frac{}{X\subseteq Y,a\in X\vdash a\in Y}\subseteq 7
\]
Note these all look good! NOW WE HAVE TO DO THE OTHER DIRECTION BECAUSE I
GUESSED WRONG.
\begin{prooftree}
    \AxiomC{$\Gamma_1,f(X,Y)\in X\vdash f(X,Y)\in Y$}
    \RightLabel{$\rightarrow R$}
    \UnaryInfC{$\Gamma_1\vdash f(X,Y)\in X\rightarrow f(X,Y)\in Y$}
    \RightLabel{$\forall R'$}
    \UnaryInfC{$\Gamma_1\vdash (\forall a.a\in X\rightarrow a\in Y)$}
        \AxiomC{$\Gamma_2,X\subseteq Y\vdash\Delta$}
    \RightLabel{$\rightarrow L$}
    \BinaryInfC{$\Gamma_1,\Gamma_2,(\forall a.a\in X\rightarrow a\in Y)\rightarrow X\subseteq Y\vdash\Delta$}
\end{prooftree}
Weird. If we instantiate $\Gamma_1=f(X,Y)\in Y$ we can close the left leaf with
reflexivity but we have an unused premise. Oh well. We forge on.
\[
    \frac{\Gamma_2,X\subseteq Y\vdash\Delta}{f(X,Y)\in
    Y,\Gamma_2\vdash\Delta}\subseteq 8*
    \quad\quad
    \frac{\Gamma_1,f(X,Y)\in X\vdash f(X,Y)\in Y}{\Gamma_1\vdash X\subseteq
    Y}\subseteq 9
\]

\[
    \frac{}{f(X,Y)\in Y\vdash X\subseteq Y}\subseteq 10*
\]
Again I don't know how we're supposed to use this Skolem stuff in a proof.

Now we do the union rule. Both ways I guess.
\begin{prooftree}
    \AxiomC{$\Gamma_1\vdash a\in X\cup Y$}
        \AxiomC{$\Gamma_2,a\in X\vdash\Gamma/\bot$}
            \AxiomC{$\Gamma_3,a\in Y\vdash\Gamma/\bot$}
        \RightLabel{$\vee L$}
        \BinaryInfC{$\Gamma_2,\Gamma_3,a\in X\vee a\in Y\vdash\Delta/\bot$}
    \RightLabel{$\rightarrow L$}
    \BinaryInfC{$\Gamma_1,\Gamma_2,\Gamma_3,a\in X\cup Y\rightarrow a\in X\vee a\in Y\vdash\Delta$}
    \RightLabel{$\forall L'(\times 3)$}
    \UnaryInfC{$\Gamma_1,\Gamma_2,\Gamma_3,\forall X,Y,a.a\in X\cup Y\rightarrow a\in X\vee a\in Y\vdash\Delta$}
\end{prooftree}
I actually so do not want to write these. Imagine them. Now the other way.
\begin{prooftree}
    \AxiomC{$\Gamma_1\vdash a\in X\vee a\in Y$}
        \AxiomC{$\Gamma_2,a\in X\cup Y\vdash\Delta$}
    \RightLabel{$\rightarrow L$}
    \BinaryInfC{$\Gamma_1,\Gamma_2,a\in X\vee a\in Y\rightarrow a\in X\cup Y\vdash\Delta$}
    \RightLabel{$\forall L'(\times 3)$}
    \UnaryInfC{$\Gamma_1,\Gamma_2,\forall X,Y,a.a\in X\vee a\in Y\rightarrow a\in X\cup Y\vdash\Delta$}
\end{prooftree}
blah blah imagine the rest (splits into two proof trees based on choice of $\vee
R$).

\subsection{The proof using AM generated rules}

Start the proof:
\begin{prooftree}
    \AxiomC{}
    \UnaryInfC{$f(X,Y,Z)\in Z\vdash^+f(X,Y,Z)\in Z$}
    \RightLabel{$\cup R$}
    \UnaryInfC{$?,f(X,Y,Z)\in Z\vdash^+.../\bot$}
        \AxiomC{$\Pi$}
        \RightLabel{$\cup R$}
        \UnaryInfC{$?,f(X,Y,Z)\in X\vdash^+.../\bot$}
    \RightLabel{$\cup L$}
    \BinaryInfC{$X\subseteq Y,f(X,Y,Z)\in X\cup Z\vdash^+ f(X,Y,Z)\in Y\cup Z$}
    \RightLabel{$\subseteq 9$}
    \UnaryInfC{$X\subseteq Y\vdash^+X\cup Z\subseteq Y\cup Z$}
    \RightLabel{$\forall R$}
    \UnaryInfC{$X\subseteq Y\vdash^+\forall Z.X\cup Z\subseteq Y\cup Z$}
    \RightLabel{$\rightarrow R$}
    \UnaryInfC{$\vdash^+X\subseteq Y\rightarrow\forall Z.X\cup Z\subseteq Y\cup Z$}
    \RightLabel{$\forall R (\times 2)$}
    \UnaryInfC{$\vdash^+\forall X.\forall Y.X\subseteq Y\rightarrow\forall
    Z.X\cup Z\subseteq Y\cup Z$}
\end{prooftree}
Supposing we closed the left branch first, by the time we reach $\Pi$ we know
that the left branch doesn't take any extra premises in its ? so $\Pi$ takes
them all. Here is the rest of the proof:
\begin{prooftree}
    \AxiomC{}
    \RightLabel{$\subseteq 7$}
    \UnaryInfC{$X\subseteq Y,f(X,Y,Z)\in X\vdash^+f(X,Y,Z)\in Y$}
\end{prooftree}
My observations while writing this:
\begin{itemize}
    \item For binary inferences, proof search cannot guess how to split the premises
    \item For the Skolem function, I had to add another free variable so is
        creation of the function postponed until application of the rule? Also
        we just treated it as a closed variable.
    \item Did not use any of the funky rules.
\end{itemize}

\subsection{Generating rules using Superdeduction}

We start with the subset rule: $\forall X,Y.(X\subseteq Y\Leftrightarrow (\forall
a.a\in X\rightarrow a\in Y))$. This is in proper format so we are good to go. To
generate the right rule, we are supposed to begin with the sequent
$\Gamma\vdash\varphi,\Delta$ but System $\mathbb{C}$ only deals with single
conclusion sequents so we remove $\Delta$.
\begin{prooftree}
    \AxiomC{$\Gamma,a\in X\vdash a\in Y$}
    \RightLabel{$\rightarrow R\quad\quad\text{where $a$ fresh}$}
    \UnaryInfC{$\Gamma\vdash a\in X\rightarrow a\in Y$}
    \RightLabel{$\forall R$}
    \UnaryInfC{$\Gamma\vdash \forall a.a\in X\rightarrow a\in Y$}
\end{prooftree}
This gives the rule
\[
\frac{\Gamma,a\in X\vdash a\in Y}{\Gamma\vdash X\subseteq Y}\subseteq
R\quad\quad\text{where $a$ is fresh}
\]
Similarly, the left rule is
\begin{prooftree}
    \AxiomC{$\Gamma_1\vdash a\in X$}
    \AxiomC{$\Gamma_2,a\in Y\vdash \Delta$}
    \RightLabel{$\rightarrow L$}
    \BinaryInfC{$\Gamma_1,\Gamma_2,a\in X\rightarrow a\in Y\vdash\Delta$}
    \RightLabel{$\forall L'\quad\quad\text{where $a$ is a closed term?}$}
    \UnaryInfC{$\Gamma_1,\Gamma_2,\forall a.a\in X\rightarrow a\in Y\vdash\Delta$}
\end{prooftree}
So the rule is
\[
\frac{\Gamma_1\vdash a\in X\quad\Gamma_2,a\in
Y\vdash\Delta}{\Gamma_1,\Gamma_2,X\subseteq Y\vdash\Delta}\subseteq
L\quad\quad\text{where $a$ is a closed term?}
\]

Similarly, we generate rules for union: $\forall X,Y,a.(a\in X\cup
Y\Leftrightarrow a\in X\vee a\in Y$. The right rule:
\begin{prooftree}
    \AxiomC{$\Gamma\vdash a\in X$}
    \RightLabel{$\vee R$}
    \UnaryInfC{$\Gamma\vdash a\in X\vee a\in Y$}
\end{prooftree}
and of course, the equivalent tree with the right-hand side. This time we get
two rules:
\[
\frac{\Gamma\vdash a\in X}{\Gamma\vdash a\in X\cup Y}\cup R1
\quad\quad
\frac{\Gamma\vdash a\in Y}{\Gamma\vdash a\in X\cup Y}\cup R2
\]
And the left rule:
\begin{prooftree}
    \AxiomC{$\Gamma_1,a\in X\vdash\Delta/\bot$}
    \AxiomC{$\Gamma_2,a\in Y\vdash\Delta/\bot$}
    \RightLabel{$\vee L$}
    \BinaryInfC{$\Gamma_1,\Gamma_2,a\in X\vee a\in Y\vdash\Delta/\bot$}
\end{prooftree}
This generates the rule
\[
\frac{\Gamma_1,a\in X\vdash\Delta/\bot\quad\Gamma_2,a\in Y\vdash\Delta/\bot}
     {\Gamma_1,\Gamma_2,a\in X\cup Y\vdash\Delta/\bot}\cup L
\]

\subsection{The proof using Superdeduction generated rules}

\begin{prooftree}
    \AxiomC{}
    \UnaryInfC{$a\in Z\vdash^+a\in Z$}
    \RightLabel{$\cup R2$}
    \UnaryInfC{$?,a\in Z\vdash^+.../\bot$}
        \AxiomC{}
        \UnaryInfC{$?\vdash a\in X$}
            \AxiomC{}
            \UnaryInfC{$?,a\in Y\vdash a\in Y$}
        \RightLabel{$\subseteq L$}
        \BinaryInfC{$X\subseteq Y,a\in X\vdash a\in Y$}
        \RightLabel{$\cup R1$}
        \UnaryInfC{$?,a\in X\vdash^+.../\bot$}
    \RightLabel{$\cup L$}
    \BinaryInfC{$X\subseteq Y,a\in X\cup Z\vdash^+ a\in Y\cup Z$}
    \RightLabel{$\subseteq R$}
    \UnaryInfC{$X\subseteq Y\vdash^+X\cup Z\subseteq Y\cup Z$}
    \RightLabel{$\forall R$}
    \UnaryInfC{$X\subseteq Y\vdash^+\forall Z.X\cup Z\subseteq Y\cup Z$}
    \RightLabel{$\rightarrow R$}
    \UnaryInfC{$\vdash^+X\subseteq Y\rightarrow\forall Z.X\cup Z\subseteq Y\cup Z$}
    \RightLabel{$\forall R (\times 2)$}
    \UnaryInfC{$\vdash^+\forall X.\forall Y.X\subseteq Y\rightarrow\forall
    Z.X\cup Z\subseteq Y\cup Z$}
\end{prooftree}

Well look at that, the proof is a little more complicated (look at the right
branch).

\subsection{The proof using regular embedded rules}

Let $H_\subseteq=\forall X,Y.(X\subseteq Y\Leftrightarrow (\forall a.a\in X\rightarrow
a\in Y))$ and $H_\cup=\forall X,Y,a.(a\in X\cup Y\Leftrightarrow a\in X\vee a\in
Y$. Also I'm going to cut some corners namely with the picking of a direction of
$\Leftrightarrow$.

\begin{prooftree}
    \AxiomC{}
    \UnaryInfC{$X\subseteq Y\vdash X\subseteq Y$}
        \AxiomC{$\Pi$}
    \BinaryInfC{$H_\cup,H_\subseteq(X\cup Z,Y\cup Z),H_\subseteq(X,Y),X\subseteq Y,a\in X\cup Z\vdash a\in Y\cup Z$}
    \RightLabel{$\forall L(\times 2)$}
    \UnaryInfC{$H_\subseteq,H_\cup,X\subseteq Y\vdash X\cup Z\subseteq Y\cup Z$}
    \RightLabel{$\forall R$}
    \UnaryInfC{$H_\subseteq,H_\cup,X\subseteq Y\vdash\forall Z.X\cup Z\subseteq Y\cup Z$}
    \RightLabel{$\rightarrow R$}
    \UnaryInfC{$H_\subseteq,H_\cup\vdashX\subseteq Y\rightarrow\forall Z.X\cup Z\subseteq Y\cup Z$}
    \RightLabel{$\forall R (\times 2)$}
    \UnaryInfC{$H_\subseteq,H_\cup\vdash\forall X.\forall Y.X\subseteq
    Y\rightarrow\forall Z.X\cup Z\subseteq Y\cup Z$}
\end{prooftree}

Wow this sucks. Now $\Pi$ is
\begin{prooftree}
    \AxiomC{$\Omega$}
        \AxiomC{}
        \UnaryInfC{$X\cup Z\subseteq Y\cup Z\vdash X\cup Z\subseteq Y\cup Z$}
    \RightLabel{$\rightarrow L$}
    \BinaryInfC{$H_\cup,H_\subseteq(X\cup Z,Y\cup Z),\forall a.a\in X\rightarrow
    a\in Y\vdash X\cup Y\subseteq Y\cup Z$}
\end{prooftree}

Almost done. $\Omega$ is
\begin{prooftree}
    \AxiomC{}
    \UnaryInfC{$a\in X\cup Z\vdash a\in X\cup Z$}
        \AxiomC{you get the idea}
        \UnaryInfC{$H_\cup(Y,Z),a\in X\rightarrow a\in Y,a\in X\vee a\in Z\vdash
        a\in Y\cup Z$}
    \RightLabel{$\rightarrow L$}
    \BinaryInfC{$H_\cup(X,Z),H_\cup(Y,Z),a\in X\rightarrow a\in Y,a\in X\cup Z\vdash
    a\in Y\cup Z$}
    \RightLabel{$\rightarrow R$}
    \UnaryInfC{$H_\cup(X,Z),H_\cup(Y,Z),a\in X\rightarrow a\in Y\vdash a\in
    X\cup Z\rightarrow a\in Y\cup Z$}
    \RightLabel{$\forall L (\times 3)$}
    \UnaryInfC{$H_\cup,\forall a.a\in X\rightarrow a\in Y\vdash a\in
    X\cup Z\rightarrow a\in Y\cup Z$}
    \RightLabel{$\forall R$}
    \UnaryInfC{$H_\cup,\forall a.a\in X\rightarrow a\in Y\vdash \forall a.a\in
    X\cup Z\rightarrow a\in Y\cup Z$}
\end{prooftree}

\subsection{Now using the base Atomic Metadeduction rules}

First, subset:
\begin{prooftree}
    \AxiomC{$\Gamma\vdash X\subseteq Y,\Delta$}
        \AxiomC{$\Gamma\vdash a\in X,\Delta$}
            \AxiomC{$\Gamma,a\in Y\vdash\Delta$}
        \RightLabel{$\rightarrow L$}
        \BinaryInfC{$\Gamma,a\in X\rightarrow a\in Y\vdash\Delta$}
        \RightLabel{$\forall L\quad\quad\text{$a$ closed}$}
        \UnaryInfC{$\Gamma,\forall a.a\in X\rightarrow a\in Y\vdash\Delta$}
    \RightLabel{$\rightarrow L$}
    \BinaryInfC{$\Gamma,X\subseteq Y\rightarrow (\forall
    a.a\in X\rightarrow a\in Y)\vdash \Delta$}
\end{prooftree}
Imagine the rules because I'm tired of typing. They look like the ones for
$\mathbb{C}$ but there is always a $\Delta$ in the conclusions of each sequent
and the premises on the bottom sequent are duplicated in the premises of the
top ones.
\begin{prooftree}
    \AxiomC{$\Gamma,f(X,Y)\in X\vdash f(X,Y)\in Y,\Delta$}
    \RightLabel{$\rightarrow R$}
    \UnaryInfC{$\Gamma\vdash f(X,Y)\in X\rightarrow f(X,Y)\in Y,\Delta$}
    \RightLabel{$\forall R$}
    \UnaryInfC{$\Gamma\vdash (\forall a.a\in X\rightarrow a\in Y),\Delta$}
        \AxiomC{$\Gamma,X\subseteq Y\vdash\Delta$}
    \RightLabel{$\rightarrow L$}
    \BinaryInfC{$\Gamma,(\forall a.a\in X\rightarrow a\in Y)\rightarrow X\subseteq Y\vdash\Delta$}
\end{prooftree}
Imagine the rules again.

Now we do the union rule. Both ways I guess.
\begin{prooftree}
    \AxiomC{$\Gamma\vdash a\in X\cup Y,\Delta$}
        \AxiomC{$\Gamma,a\in X\vdash\Gamma$}
            \AxiomC{$\Gamma,a\in Y\vdash\Gamma$}
        \RightLabel{$\vee L$}
        \BinaryInfC{$\Gamma,a\in X\vee a\in Y\vdash\Delta$}
    \RightLabel{$\rightarrow L$}
    \BinaryInfC{$\Gamma,a\in X\cup Y\rightarrow a\in X\vee a\in Y\vdash\Delta$}
    \RightLabel{$\forall L(\times 3)$}
    \UnaryInfC{$\Gamma,\forall X,Y,a.a\in X\cup Y\rightarrow a\in X\vee a\in Y\vdash\Delta$}
\end{prooftree}
I actually so do not want to write these. Imagine them. Now the other way.
\begin{prooftree}
    \AxiomC{$\Gamma\vdash a\in X, a\in Y,\Delta$}
    \RightLabel{$\vee L$}
    \UnaryInfC{$\Gamma\vdash a\in X\vee a\in Y,\Delta$}
        \AxiomC{$\Gamma,a\in X\cup Y\vdash\Delta$}
    \RightLabel{$\rightarrow L$}
    \BinaryInfC{$\Gamma,a\in X\vee a\in Y\rightarrow a\in X\cup Y\vdash\Delta$}
    \RightLabel{$\forall L(\times 3)$}
    \UnaryInfC{$\Gamma,\forall X,Y,a.a\in X\vee a\in Y\rightarrow a\in X\cup Y\vdash\Delta$}
\end{prooftree}

The proof becomes
\begin{prooftree}
    \AxiomC{}
    \UnaryInfC{$f(X,Y,f())\in f()\vdash^+f(X,Y,f())\in f()$}
    \RightLabel{$\cup R$}
    \UnaryInfC{$X\subseteq Y,f(X,Y,f())\in f()\vdash^+...$}
        \AxiomC{$\Pi$}
        \RightLabel{$\cup R$}
        \UnaryInfC{$X\subseteq Y,f(X,Y,f())\in X\vdash^+...$}
    \RightLabel{$\cup L$}
    \BinaryInfC{$X\subseteq Y,f(X,Y,f())\in X\cup f()\vdash^+ f(X,Y,f())\in Y\cup f()$}
    \RightLabel{$\subseteq 9$}
    \UnaryInfC{$X\subseteq Y\vdash^+X\cup f()\subseteq Y\cup f()$}
    \RightLabel{$\forall R$}
    \UnaryInfC{$X\subseteq Y\vdash^+\forall Z.X\cup Z\subseteq Y\cup Z$}
    \RightLabel{$\rightarrow R$}
    \UnaryInfC{$\vdash^+X\subseteq Y\rightarrow\forall Z.X\cup Z\subseteq Y\cup Z$}
    \RightLabel{$\forall R (\times 2)$}
    \UnaryInfC{$\vdash^+\forall X.\forall Y.X\subseteq Y\rightarrow\forall
    Z.X\cup Z\subseteq Y\cup Z$}
\end{prooftree}
and then
\begin{prooftree}
    \AxiomC{}
    \RightLabel{$\subseteq 7$}
    \UnaryInfC{$X\subseteq Y,f(X,Y,Z)\in X\vdash^+f(X,Y,Z)\in Y$}
\end{prooftree}

\subsection{A slightly more involved proof}

Now what if we want to prove $A\subseteq B,B\subseteq C\rightarrow A\cup
Z\subseteq C\cup Z$? In a human proof, there are two approaches, seemingly
equally good:
\begin{enumerate}
    \item By transitivity, $A\subseteq C$. Then from our previous result, $A\cup
        Z\subseteq C\cup Z$.
    \item $A\cup Z\subseteq B\cup Z\subseteq C\cup Z$.
\end{enumerate}
I don't know how to formalize the second.

\end{document}
